\documentclass[a4paper,12pt]{article} %размер бумаги устанавливаем А4, шрифт 12пунктов
\usepackage[T2A]{fontenc}
\usepackage[utf8]{inputenc}	%кодировка
\usepackage[english,russian]{babel}%используем русский и английский языки с переносами
\usepackage{amssymb,amsfonts,amsmath,cite,enumerate,float,indentfirst} %пакеты расширений
\usepackage{graphicx} %вставка графики
\usepackage{sectsty,textcase}
\usepackage{graphics}

\graphicspath{{images/}}%путь к рисункам
\linespread{1.5}
\floatname{Times New Roman}{}

\makeatletter
\renewcommand{\@biblabel}[1]{#1.} % Заменяем библиографию с квадратных скобок на точку:
\renewcommand{\labelitemi}{--} % Заменяем стандартный символ маркированного списка

% Переопределение команды секции
\renewcommand{\section}{\@startsection{section}{1}%
{\parindent}{3.25ex plus 1ex minus .2ex}%
{1.5ex plus .2ex}{\bfseries\large\centering\MakeTextUppercase}}

% Переопределение команды подсекции
\renewcommand{\subsection}{\@startsection{subsection}{2}%
{\parindent}{3.25ex plus 1ex minus .2ex}%
{1.5ex plus .2ex}{\bfseries\centering\MakeTextUppercase}}
\makeatother

\usepackage{geometry} % Меняем поля страницы
\geometry{left=3cm}% левое поле
\geometry{right=1.5cm}% правое поле
\geometry{top=2cm}% верхнее поле
\geometry{bottom=2cm}% нижнее поле

\renewcommand{\theenumi}{\arabic{enumi}}% Меняем везде перечисления на цифра.цифра
\renewcommand{\labelenumi}{\arabic{enumi}}% Меняем везде перечисления на цифра.цифра
\renewcommand{\theenumii}{.\arabic{enumii}}% Меняем везде перечисления на цифра.цифра
\renewcommand{\labelenumii}{\arabic{enumi}.\arabic{enumii}.}% Меняем везде перечисления на цифра.цифра
\renewcommand{\theenumiii}{.\arabic{enumiii}}% Меняем везде перечисления на цифра.цифра
\renewcommand{\labelenumiii}{\arabic{enumi}.\arabic{enumii}.\arabic{enumiii}.}% Меняем везде перечисления на цифра.цифра

\newcommand{\imgh}[3]{\begin{figure}[h]\center{\includegraphics[width=#1]{#2}}\caption{#3}\label{ris:#2}\end{figure}} % команда для добавления изображений

\newcommand{\anonsection}[1]{\section*{#1}\addcontentsline{toc}{section}{#1}} % секция для введения и заключения

\begin{document}
\pagestyle{empty}
\begin{titlepage}
\newpage

\begin{center}
НАЗВАНИЕ УЧЕБНОГО ЗАВЕДЕНИЯ \\
\end{center}

\vspace{8em}

\begin{center}
\Large Название кафедры \\ 
\end{center}

\vspace{2em}

\begin{center}
\textsc{\textbf{Название темы работы \linebreak длинное очень, набранное в \LaTeX{}}}
\end{center}

\vspace{6em}



\newbox{\lbox}
\savebox{\lbox}{\hbox{Фамилия И.О.}}
\newlength{\maxl}
\setlength{\maxl}{\wd\lbox}
\hfill\parbox{11cm}{
\hspace*{5cm}\hspace*{-5cm}Студент:\hfill\hbox to\maxl{Фамилия И.О.\hfill}\\
\hspace*{5cm}\hspace*{-5cm}Преподаватель:\hfill\hbox to\maxl{Фамилия И.О.}\\
\\
\hspace*{5cm}\hspace*{-5cm}Группа:\hfill\hbox to\maxl{NNN}\\
}


\vspace{\fill}

\begin{center}
Город \\год
\end{center}

\end{titlepage}% титульный лист
\newpage
\tableofcontents % оглавление, которое генерируется автоматически
\newpage
\setcounter{page}{3}
\pagestyle{plain}

\anonsection{Введение}
Здесь должен быть текст введения.% введение
\newpage
\section{Работа с шаблоном}
\subsection{Списки}
Список выглядит следующим образом:
\begin{itemize}
\item раз;
\item два;
\item три.
\end{itemize}

\subsection{Картинки}
Пример вставки картинки и ссылки на нее (рис. \ref{ris:black_cat.jpg}).

\imgh{95pt}{black_cat.jpg}{Черная кошка.}

Пример вставки картинки и ссылки на нее (рис. \ref{ris:white_cat.jpg}).

\imgh{190pt}{white_cat.jpg}{Белый котенок.}

\subsection{Ссылки на источники}
Ссылка на источник 1 \cite{src1}.

Ссылка на источник 2 \cite{src2}.

Ссылка на источник 3 \cite{src3}.% основной текст документа
\newpage
\anonsection{Заключение}
Текст заключения.% заключение
\newpage
\begin{thebibliography}{3}
\bibitem{src1}
Какой-то источник
\bibitem{src2}
Еще один источник
\bibitem{src3}
И еще один
\end{thebibliography}

\end{document}